\chapter{General Evaluation}

This section looks at the consequences of getting a system evaluated
against one of the sets of security evaluation criteria -- TCSEC, ITSEC
and CTCPEC from the perspective of the developer, the procurer and the
end user.

\section{Consequences for the developer}

    Developers need to be aware of technical security issues, protocols
    and standards specific to the system being built, in some cases
    interpretations of the criteria, and should be aware of security
    guidelines or requirements which are not specified in the criteria
    but have being identified by a client or other stake-holder in the
    development of the system.

    All the assurance levels of the three security evaluation criteria
    require that the developer employ standard software engineering
    practices, such as system requirements (formal) specification, design, 
    implementation, testing, documentation, configuration management, 
    and (formal) verification and validation. In addition to these general software 
    engineering requirements, there are many requirements specifically 
    relating to security, such as the development of the security policy model.

    If the system is being evaluated against the TCSEC, the developer
    should be aware that the interpretations of the TCSEC, the ``Rainbow
    Series'', need to be consulted for specific application type.
    Interpretations are not required for the ITSEC or the CTCPEC,
    as all these criteria were developed to target a greater range of
    trusted systems. If the system being evaluated against the ITSEC,
    the developer should be aware that functionality criteria are not
    specified in the standard and it is up to the developer to define
    the security functions of the system.

\section{Consequences for the development organisation}

    The organisation developing a trusted system must be familiar with the assurance levels of the
    criteria and must notify the organisation conducting the evaluation
    to which assurance level the system is being targeted at. This means
    that the evaluation will only attempt to determine if the system
    provides that level of trust or assurance. 

    The evaluation body usually also places other requirements on the
    development organisation.
    For example, in the United States, if a system is being submitted
    to the Trusted Product Evaluation Program (TPEP)~\footnote{The TPEP
    is part of the National Security Agency (NSA), which is part of the
    US Department of Defense} for evaluation against the TCSEC,
    the organisation must provide evidence that
    the system has a legitimate market in the United States.

    Although evaluation organisation such as TPEP do not charge the
    development organisation a fee for an evaluation, there should be 
    an awareness that
    developing a system which satisfies any of the assurance levels of
    criteria such as TCSEC, ITSEC and CTCPEC results in very high
    development costs, especially at the higher levels of assurance.

\section{Consequences for the end user}

    A user or a purchaser of a secure system needs to aware of a number
    of issues.
    \begin{itemize}
        \item What are the security requirements?
        \item Is there familiarity with the security requirements
              of the evaluation criteria which the product being 
              considered has being evaluated against?
        \item With what level of assurance in the relevant evaluation
              criteria does the user's security requirement correspond
              to?
    \end{itemize}

    The user must also be aware of the secure installation, startup and
    operation of the system in order for the security requirements to be
    fulfilled.

    In general, a system which has been successfully evaluated against
    one of the evaluation criteria provides the user with a high degree
    of confidence that their is assurance that the system provides the
    level of assurance which it has been awarded.









