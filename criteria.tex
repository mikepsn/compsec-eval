\chapter{Comparison Characteristics}

    \section{Method}
    The method followed in this comparison is as follows:
    \begin{itemize}
        \item A number of computer security standards relating to
              the evaluation of trusted systems were chosen.
        \item A list of evaluation characteristics were selected to
              form the basis of the comparison.
        \item The chosen documents were then summarised 
              against the comparison characteristics.
    \end{itemize}
    This method was chosen because by evaluating the standards
    against a number of characteristics, then provided we can extract some
    useful information from the characteristics, we can also draw conclusions
    about the standards.
    The comparison of the security standards was based on the
    general characteristics discussed in this section. 
    The criteria are divided into three general categories:
        \begin{description}
            \item[General Comparison:] 
                A general comparison provides us with information
                regarding the scope of each of the documents, the types
                of criteria specified, the types of computer systems
                targeted, the criteria for the different levels of trust
                and information on the reliance on separate
                interpretations of the standards.
            \item[Security Functionality:] 
                This is a comparison of the
                standards based on the different aspects of functional
                security criteria. This allows us to look at how the standards
                address different issues relating to security functionality.
            \item[Functionality Assurance:] A comparison of the
                security standards based on various system engineering
                practices which must be put in place to assure that the
                security functionality has been successfully implemented
                and in fact provide operational security at the required
                level of trust. This tells us the type of issues which the
                developer must consider when designing and building a system
                which needs to be evaluated against one of the standards.
        \end{description}

    Each category is further divided up into a number of characteristics. A
    description of each the characteristics is given in this section.

    The choice of the the characteristics on which the comparison is
    made is not meant to be extensive. Rather, the characteristics were
    chosen as general areas in which organisations developing trusted
    systems need to be aware of and then a comparison of how each of the
    documents deals with these characteristics is made.

    \section{Organisation}

        \subsection{Structure}
            The overall structure of the security evaluation criteria
            documents is compared, including the purpose and intended
            audience of each document. The comparison on structure is an
            overall general description of the similarities and
            differences between the documents.

            This characteristic was chosen because it provides us with 
            information regarding the scope of the documents and type of security
            issues that are addressed. This information is relevant to both
            developers and those involved in evaluating systems.

        \subsection{Levels of Trust}
            A level of trust is a measure of the security functionality 
            that a system provides, and the assurance of security that the
            system gives.
            Each standard specifies a number of levels of trust~\footnote{Also known
            as assurance levels, evaluation levels and classes}.
            A computer product
            which is evaluated against a set of evaluation criteria is
            given a rating corresponding to a level of trust of
            satisfying the requirements of that level.

            This characteristic was chosen because levels of trust are the primary
            indicator that establish what level of security a system provides.
            Organisations make decisions on what evaluated systems should be purchased
            according the level at which the system is evaluated at. This allows the
            organisation to look at the standards and see what level of trust corresponds
            to their requirements. 

            This characteristic allows us too look at what is specified for each level
            of trust in each of the security evaluation criteria, and what security issues
            are addressed at each level.

    \section{Security Functionality}
        The three characteristics of accountability, access control and audit were chosen
        as a basis for comparing the security functionality criteria of the three
        security standards. Although this is far from being a complete list of all the 
        security functions which a system can provide, the purpose was to select 
        only a few of the most important security functions. By looking at three of
        the most important and widely implemented security functions, we are able
        to assess how each of the security evaluation criteria addresses security functionality
        at a general level.

        \subsection{Accountability --- Identification and Authentication}
            Accountability in computer security consists of two main
            areas:
                \begin{itemize}
                    \item Identification 
                    \item Authentication
                \end{itemize}
            Identification deals with identifying a user, and the files,
            processes, actions and access to system resources 
            associated with the user.
            Authentication involves verifying the identity of a user or
            process so that it can be decided which resources may be
            accessed by the user.

        \subsection{Access Control}
            Discretionary Access Control (DAC), refers to information
            in a computer system which is by default not freely
            available, but access may be granted to other users at the
            discretion of the owner of the information.

            Mandatory Access Control (MAC) refers to the access of
            information through classification and labeling and
            different levels, accessed by users with authorisation at
            those classifications.

        \subsection{Audit}
            Auditing in secure computer systems serves two main
            purposes. Firstly it allows the monitoring of a system's
            operation (through mechanisms such as audit logs), to
            identify security breaches and to facilitate corrective
            action. Secondly it is used in functionality
            assurance (see section~\ref{functionality-assurance}) by certifying
            that a system meets certain security requirements.

    \section{Security Functionality Assurance}
    \label{functionality-assurance}
    Security functionality assurance deals with the practices which 
    must be put in place by a developer for a system to be evaluated at
    a particular assurance level. Therefore all the characteristics chosen in this
    category for the comparison deal with the different system engineering
    practices associated with the system development life cycle.

    The characteristics selected are security policy, system design,
    implementation, security testing, security documentation, and
    configuration control. The information provided by the comparison is of
    importance to developers as it indicates the kind of engineering practices
    which have to be put in place when developing systems to be evaluated
    against one the standards.
    
        \subsection{Security Policy}
            A security policy is a set of requirements with respect to the 
            security function that a system provides. From the security
            policy a security policy model is developed, which contains
            detailed descriptions of a system's security functions.
            The security policy model may be described using a 
            informal, semi-formal or formal specification style.
            
        \subsection{System Design}
            Systems requiring high levels of trust must be designed to
            ensure that the the system's security policy is adequately
            enforced. The system design characteristic will look at what
            the each of the evaluation criteria specify as requirements
            for the design of a secure system to assure the system's
            security functionality;

            This includes the requirements for formal designs at both
            the architectural and detailed level, the design
            methodologies used,  and traceability and mapping between
            the design and the requirements specification.

        \subsection{Implementation}
            The implementation requirements for a system in each of the
            evaluation criteria look at issues such as the choice and use of 
            well defined programming languages, coding standards and
            compliance with these standards, the choice of development
            tools such as compilers, the provision of source code for
            evaluation, the mapping between the source code and the
            detailed design, and the specification of implementation
            specific options.

        \subsection{Security Testing}
            Although some form of testing is present in all system
            engineering processes, when developing trusted and secure
            systems it is of very important to ensure that the system 
            is tested adequately so that the user of the system has some
            level of confidence that the system's security policy is
            being satisfied.

            Issues such as the type of testing required for different
            levels of trust, test documentation such as test plans, test
            results, and justification that the testing is sufficient.

        \subsection{Security Documentation}
            Trusted systems must be delivered with documentation to
            adequately inform both the user and system administrator
            about the security features of the system and the secure
            operation and administration of the system.

        \subsection{Configuration Management}
            Trusted systems must be developed under approved and
            reliable configuration management system, so that all source
            code, object code, and all documentation (requirements,
            design, testing, user and security) are under revision
            control and are consistent.

