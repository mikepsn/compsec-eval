\chapter{Computer Security Evaluation Criteria}

    \section{Choice of Computer Security Evaluation Criteria}
        The are many computer security evaluation criteria from countries
        all around the world. Only a subset of the many available ones have
        be chosen to be compared. 

        Although many of the security evaluation criteria in use are
        aimed at evaluating products for military and intelligence
        systems, they are usually have a broad scope and are just as
        applicable in many other areas as well.

        The three most influential and widely used security evaluation
        criteria were chosen for the comparison. These documents are
        the specified security standards of the United States, Europe
        and Canada. The three standards are shown below.
        \begin{itemize}
            \item Trusted Computer Security Evaluation Criteria (TCSEC) [USA]
            \item Information Technology Computer Security 
                  Evaluation Criteria (ITSEC) [Europe]
            \item Canadian Trusted Computer Product Evaluation Criteria
                  (CTCPEC) [Canada]
        \end{itemize}
        The following sections contain more detailed information on why these standards
        were chosen and other standards were omitted.

    \section{United States Security Evaluation Criteria}
        The most widely used and best known security evaluation criteria
        is the US Department of Defense ``Trusted Computer
        System Evaluation Criteria (TCSEC)''~\cite{orange}, also known as the 
        ``Orange Book''. The latest version of this document is dated
        1985 and it was included in the comparison. 

        In 1992 a draft document of a new set of security criteria known
        as the ``Federal Criteria''~\cite{federal} which was intended to eventually
        replace the TCSEC, was developed . The ``Federal Criteria'' did not get beyond
        draft stage and are not widely used, and therefore were not 
        included in the comparison.

    \section{European Security Evaluation Criteria}
        There are a number of European security standards in existence.
        \begin{itemize}
            \item UK Systems Security Confidence Levels, CESG Memorandum Number 3~\cite{cesg}.
            \item UK Commercial Computer Security Centre Evaluation Levels Manual~\cite{dtiec} 
                  (Also known as ``The Green Book'').
            \item Criteria for the Evaluation of Trustworthiness of Information Technology Systems~\cite{german} 
                  (German Information Security Agency).
            \item The French ``Catalogue de Crit res Destin s valuer le Degt de Confiance des
                  Syst mes d'Information''~\cite{french}.
                  (Also known as the `Blue-White-Red'' book).
        \end{itemize}

        It was decided that work being done in Europe should be combined
        into a harmonised set of security evaluation criteria. This led
        to the development by the United Kingdom, Germany, the
        Netherlands, and France of the ``Information Technology Security
        Evaluation Criteria (ITSEC)''~\cite{itsec}. As the ITSEC document replaced
        all the other European security evaluation criteria mentioned
        above, it is the only European document selected for the comparison.

    \section{Canadian Security Evaluation Criteria}
        The most important security criteria currently in use in Canada
        was developed by the Canadian government and is known as the 
        ``Canadian Trusted Computer Product Evaluation
        Criteria''~\cite{ctcpec}  and
        was included in the comparison.

    \section{Common Criteria}
        The ``Common Criteria for Information Technology Security
        Evaluation''~\cite{cc} also known as ``CC'' are an international effort to combine
        the TCSEC, ITSEC, and CTCPEC into one common set of security
        evaluation criteria. The Common Criteria are at version 1.0, and
        are currently undergoing trial evaluations and review and hence
        were not considered in this comparison.


