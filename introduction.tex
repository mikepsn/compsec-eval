
\chapter{Introduction}

    Computer systems have been used to store and process security critical information for decades.
    Secure computing systems, once solely used by the military, are required in an ever increasing 
    range of applications. Such systems are being used in government, industry, in financial applications
    such as banking, and over the last couple of years in many Internet applications such as
    electronic commerce. Secure systems are required almost anywhere sensitive information is stored,
    processed and where protection of this information is required.

    It is therefore of great importance that the systems used in security critical applications
    conform to certain security standards. We must be able to evaluate systems against a computer 
    security standard and determine, firstly what are the security features that the system provides,
    and secondly what level of assurance do we have that the system securely provides these features.
    Considerable effort has been expended by many countries to develop information technology security
    standards. In particular, over the past decade the concepts and criteria used to evaluated 
    secure computer systems have matured in the United States, Europe and Canada.

    A computer security standard provides a set of criteria which a 
    product, such as an operating system, can be compared against 
    to show the level of security which the product
    provides. 
    However, there are number of security standards currently in use.
    The choice of standard affects how
    widely the evaluation of a product is accepted by potential customers around
    the world, especially in countries where different standards are in use. 
    
    Hence, there is
    justification in looking at and comparing a number of security
    standards. This not only involves looking at the similarities and
    differences in security criteria which each standard specifies, but
    also comparing the different levels of trust at which products are
    evaluated against. Computer security standards dealing with the
    evaluation of complete systems are known as computer security
    evaluation criteria.
   
    This paper presents the results of a comparison between comparing a number
    of the most important, widely and currently used computer security
    standards. 
    The purpose of the comparison was to determine what the consequences were
    for choosing a particular standard from the perspective of the developer,
    the development organisation and the end user. 
    The comparison was made against a number of relevant characteristics.
    These include general characteristics such as levels of trust provided
    by the standards as well
    as more specific characteristics relating to security functionality and
    assurance.
    
    In this paper we describe the selected characteristics and justify their
    choice. This is followed by a description of the security standards selected
    for the comparison and why they were chosen. The general and specific
    comparisons based on the selected characteristics are then described,
    followed by an evaluation of the consequences of using the standards and
    then a summary of the conclusions of the comparison.

