\chapter{Comparison of the Computer Security Evaluation Criteria} 

\section{Comparing Organisational Structure} 
\label{structure}

    Table~\ref{table:docstructure} shows the document structure for each of the
    security evaluation criteria being compared. 

    The first part of the TCSEC describes the 
    criteria used to evaluate computer systems. The criteria are divided into
    four divisions which are given the following names.
        \begin{itemize}
            \item Division D -- Minimal Protection
            \item Division C -- Discretionary Protection
            \item Division B -- Mandatory Protection
            \item Division A -- Verified Protection
        \end{itemize}
    Each division consists of one or more classes. 
    Each class contains a list of security functionality and security assurance
    requirements which a system must satisfy to be evaluated at that class.
    There are seven classes in the TCSEC --- A1, B3, B2, B1, C2, C1, and D, in 
    decreasing order of features and assurances.
    The requirements for a higher class are always a superset of the lower class.

    The second part of the TCSEC contains some information regarding the rationale
    behind the development of the evaluation classes. It also contains some
    guidelines on various security issues which are specified in the evaluation classes 
    such as security testing.

    The second column of table~\ref{table:docstructure} shows the document structure
    for the ITSEC. The ITSEC describes it's approach to functionality
    criteria in chapter 2, and effectiveness assurance in chapter 3.
    This is followed by a chapter on the the security assurance levels.
    There are seven levels of assurance in the ITSEC --- E6, E5, E4, E3, E2,
    E1, and E0 in decreasing order of features and assurances similar to the
    TCSEC. 

    Example functionality classes are contained in Annex A of the ITSEC, 
    and are not part of the main criteria.
    Unlike the TCSEC and the CTCPEC, the ITSEC does not define it's own
    security functionality criteria. It provides guidelines, recommendations
    and examples 
    for selecting functionality criteria for a system.

    \begin{table}[H]
    \begin{center} 
    \begin{tabular}{|>{\small \sffamily}p{5.3cm}
                    |>{\small \sffamily}p{5.3cm}
                    |>{\small \sffamily}p{5.3cm}|} \hline
        \bfseries{TCSEC} & \bfseries{ITSEC} & \bfseries{CTCPEC} \\ \hline \hline
        \begin{enumerate} 
            \item Introduction
            \item Part I : The Criteria
                \begin{itemize}
                    \item Division D
                        \begin{itemize}
                            \item Class D
                        \end{itemize}
                    \item Division C
                        \begin{itemize}
                            \item Class C1
                            \item Class C2
                        \end{itemize}
                    \item Division B
                        \begin{itemize}
                            \item Class B1
                            \item Class B2
                            \item Class B3
                        \end{itemize}
                    \item Division A
                        \begin{itemize}
                            \item Class A1
                        \end{itemize}
                \end{itemize}
            \item Part II : Rationale and Guidelines
                \begin{itemize}
                    \item Control objectives for trusted computer systems
                    \item Rationale behind the evaluation classes
                    \item The relationship between policy and the criteria
                    \item A guideline on covert channels
                    \item A guideline on configuring mandatory access control features
                    \item A guideline on security testing
                \end{itemize}
            \item Appendices
                \begin{itemize}
                    \item A. Commercial Product Evaluation Process
                    \item B. Summary of Evaluation Criteria Divisions
                    \item C. Summary of Evaluation Criteria Classes
                    \item D. Requirement Directory
                \end{itemize}
            \item Glossary
            \item References
        \end{enumerate} 
        &  
        \begin{enumerate}
            \item Introduction
            \item Functionality
            \item Assurance -- Effectiveness
            \item Assurance -- Correctness
                \begin{itemize}
                    \item Level E0
                    \item Level E1
                    \item Level E2
                    \item Level E3
                    \item Level E4
                    \item Level E5
                    \item Level E6
                \end{itemize}
            \item Results of Evaluation
            \item Glossary and References
            \item Annex A -- Example Functionality Classes
                \begin{itemize}
                    \item F-C1
                    \item F-C2
                    \item F-B1
                    \item F-B2
                    \item F-B3
                    \item F-IN
                    \item F-AV
                    \item F-DI
                    \item F-DC
                    \item F-DX
                \end{itemize}
            \item Annex B -- The Claims Language
        \end{enumerate}
        &
        \begin{enumerate}
            \item Introduction
            \item Confidentiality Criteria
            \item Integrity Criteria
            \item Availability Criteria
            \item Accountability Criteria
            \item Assurance Criteria
                \begin{itemize}
                    \item T0 -- non-compliant
                    \item Level T-1
                    \item Level T-2
                    \item Level T-3
                    \item Level T-4
                    \item Level T-5
                    \item Level T-6
                    \item Level T-7
                \end{itemize}
            \item Definitions
            \item Bibliography
            \item Appendices
                \begin{itemize}
                    \item A. Technical Rationale
                    \item B. Constraints
                    \item C. Fundamentals
                    \item D. Concepts
                    \item E. Guide to Object Mediation
                    \item F. Guide to Confidentiality
                    \item G. Guide to Integrity
                    \item H. Guide to Availability
                    \item I. Guide to Accountability
                    \item J. Guide to Assurance
                    \item K. Implementing Services via Cryptography
                    \item L. Government Security Policy and Standards
                    \item M. Security Functionality Profiles
                \end{itemize}
        \end{enumerate}
        \\ \hline
    \end{tabular}
    \end{center} 
    \caption{Document Structure of Computer Security Evaluation Criteria} \label{table:docstructure}
    \end{table}

    The evaluation criteria of the CTCPEC are divided into those dealing with security functionality and those dealing
    with security assurance. 
    The CTCPEC contains individual chapters describing confidentiality criteria, integrity criteria, 
    availability criteria, accountability criteria and assurance criteria. 
    There are eight levels of assurance in the ITSEC --- T7, T6, T5, T4, RT, T2, T1, and T0 in 
    decreasing order of assurance.

\section{Comparing Levels of Trust}
This section looks at the different levels of trust specified by each of the documents.
A level of trust is metric used to measure the level of security features and assurance
a system provides.

\subsection{Classes of the TCSEC}
\label{tcsec-classes}

    Table~\ref{table:tcsec-class-table} summarises the requirements for each of the 
    classes of the TCSEC. Each class in the TCSEC addresses four different
    areas.

    \subsubsection{Security Policy}
    All the classes above class D, in the TCSEC require the description of a system security policy.
    The security policy covers functionality such discretionary and mandatory access control and labelling.
    The specification of a security policy model is required in some of the classes, with the higher
    classes requiring use of semi-formal or formal methodologies, for the specification of the model
    and for the verification of the system design.

    \subsubsection{Accountability}
    The accountability requirements of the TCSEC classes deal with issues such as identification,
    authentication, trusted path and auditing. The higher the class, the stricter the requirements
    for identification, authentication, and auditing.

    \subsubsection{Assurance}
    The TCSEC classes address operational assurance (covert channels, trusted recovery)
    and life cycle assurance (requirements, design, implementation, testing, formal specification
    and verification).

    \subsubsection{Documentation}
    All TCSEC classes require the provision of security documentation for the user and
    the system administrator. The higher classes require the provision of test and design
    documentation.

    \begin{table}[H]
        \begin{center}
        \begin{tabular}{|>{\sffamily}p{1cm}|>{\sffamily}p{12cm}|} \hline
            \multicolumn{2}{|c|}{\bfseries{\sffamily{TCSEC}}} \\ \hline
            \bfseries{Level} & \bfseries{Description} \\ \hline
            D  & Minimal Protection \newline
                 Inadequate assurance
                 \\ \hline
            C1 & Discretionary Security Protection \newline
                 Separation between users and data,
                 enforcement of access limitations for individual users.
                 Use of discretionary access control, design
                 documentation, identification and authentication,
                 secure system architecture and integrity,
                 security testing, security documentation for
                 users and administrators and test documentation.
                 \\ \hline
            C2 & Controlled Access Protection \newline
                    C1 requirements and stricter discretionary
                    access control. Accountability of all users
                    through procedures such as auditing.
                    \\ \hline
            B1 & Labelled Security Protection \newline
                    C2 requirements, and informal security policy
                    model specification, data labelling according
                    to specification, and mandatory access control.
                    Formal or informal design specification and
                    verification.
                    \\ \hline
            B2 & Structured Protection \newline
                    B1 requirements and a formal security policy model,
                    extended discretionary and mandatory access control
                    enforcement. Addressing of covert channels, 
                    separation of security critical components,
                    strict configuration management and strong
                    authentication mechanisms.
                    \\ \hline
            B3 & Security Domains \newline
                    B2 requirements and significant engineering
                    to minimize system's complexity, expanded audit and
                    security administration mechanisms, and system
                    recovery procedures.
                    \\ \hline
            A1 & Verified Design \newline
                    B3 requirements and use of formal design specification and verification
                    techniques throughout whole life of the system,
                    starting from a formal security policy model, a
                    formal top level specification of the design and a
                    formal detailed design. Support for strict
                    configuration management controls and secure
                    distribution of the system.
                    \\ \hline
        \end{tabular}
        \end{center}
    \caption{Summary of the TCSEC assurance levels} \label{table:tcsec-class-table}
    \end{table}

\subsection{Assurance Levels of the ITSEC}
\label{itsec-classes}
    Table~\ref{table:itsec-class-table} summarises the requirements for the different levels of 
    assurance of the ITSEC.
    The requirements for the ITSEC assurance levels are divided into a number clearly
    defined areas.
    There are four areas dealing with the development
    and the operation of the system. Each area is divided into smaller
    categories. At each level of assurance each category has a number of requirements which a system
    must satisfy for it to be evaluated at that level.

    \subsubsection{Development Process}
        \begin{itemize}
            \item Requirements Specification
            \item Architectural Design
            \item Detailed Design
            \item Implementation
        \end{itemize}

    \subsubsection{Development Environment}
        \begin{itemize}
            \item Configuration Control
            \item Programming Languages and Compilers
            \item Developer's Security
        \end{itemize}

    \subsubsection{Operational Documentation}
        \begin{itemize}
            \item User Documentation
            \item Administrator Documentation
        \end{itemize}

    \subsubsection{Operational Environment}
        \begin{itemize}
            \item Delivery and Configuration
            \item Start-up and Operation
        \end{itemize}

    \begin{table}[H]
        \begin{center}
        \begin{tabular}{|>{\sffamily}p{1cm}|>{\sffamily}p{12cm}|} \hline
            \multicolumn{2}{|c|}{\bfseries{\sffamily{ITSEC}}} \\ \hline
            \bfseries{Level} & \bfseries{Description} \\ \hline
            E0 & Inadequate Assurance \\ \hline
            E1 & Provision of a security policy, 
                 informal architectural design, and security testing
                 to show the system satisfies it's security policy.
                 \\ \hline
            E2 & E1 requirements and an informal detailed design,
                 evidence of functional testing, use of a configuration
                 control system, and use of an approved distribution
                 procedure.
                 \\ \hline
            E3 & E2 requirements and the provision of source code
                 and/or hardware corresponding to the security 
                 mechanisms. Evidence of testing those security
                 mechanisms. 
                 \\ \hline
            E4 & E3 requirements and formal security policy model
                    supporting the system's security policy. 
                    Security enforcing functions, architectural design
                    and detailed design specified in a semi-formal
                    style.
                    \\ \hline
            E5 & E4 requirements and a close correspondence between the 
                    detailed design and source code and/or hardware.
                    \\ \hline
            E6 & E5 requirements and formal specification of the of 
                    security enforcing functions and architectural
                    design, consistent with the formal security
                    policy model.
                    \\ \hline
        \end{tabular}
        \end{center} 
    \caption{Summary of the ITSEC assurance levels} \label{table:itsec-class-table}
    \end{table}

    \subsubsection{Security Functionality Criteria}
    Unlike the TCSEC and the CTCPEC, the ITSEC does not specify
    any functionality criteria. It is the responsibility of the developer to provide a specification of the
    security enforcing functions which the system provides. Annex A of the ITSEC contains example
    functionality classes which the developer can use as a guide.
    Although the selection of arbitrary security functionality
    in a system is allowed by the ITSEC, it is recommended that the system's security functions
    are grouped into eight generic categories.

    \begin{itemize}
        \item Identification and Authentication
        \item Access Control
        \item Audit
        \item Object Reuse
        \item Accuracy
        \item Reliability of Service
        \item Data Exchange
    \end{itemize}

\subsection{Assurance Levels and Security Ratings of the CTCPEC}
\label{ctcpec-classes}

    Table~\ref{table:ctcpec-levels} summarises the requirements for the assurance levels of the CTCPEC.
    As can be seen from the table, there is an emphasis on semi-formal and formal specifications
    at the higher classes for the security policy model, and the architectural and detailed design.
    There is also an emphasis on verification, through the requirements that there is a mapping
    or tracing between the different phases of the system development.
    Each assurance level in the CTCPEC addresses six different areas.

    \subsubsection{Architecture}
    This area covers requirements relating to a system's overall architecture and the enforcement
    of a system's security policy.

    \subsubsection{Development Environment}
    This area covers requirements such as the choice of development process, and configuration management.

    \subsubsection{Development Evidence}
    This area covers the requirements which show evidence of the product being developed.
    This includes requirements for a functional specification, architectural and detailed design
    and implementation.
    
    \subsubsection{Operational Environment}
    The operational environment requirements include the secure distribution, installation,
    startup and operation of a trusted system.

    \subsubsection{Security Documentation}
    The security documentation requirements include the requirements for both user and administrator
    documentation.

    \subsubsection{Security Testing}
    The security testing requirements cover issues such as security test planning,
    security test procedures, justification as to why the test coverage is sufficient,
    and evidence of security testing through the provision of test results.

    \begin{table}[H]
        \begin{center}
         \begin{tabular}{|>{\sffamily}p{1cm}|>{\sffamily}p{12cm}|} \hline
            \multicolumn{2}{|c|}{\bfseries{\sffamily{CTCPEC}}} \\ \hline
            \bfseries{Level} & \bfseries{Description} \\ \hline
            T0 & Inadequate level of assurance
                 \\ \hline
            T1 & Informal security policy, 
                 informal architectural design,
                 informal detailed design of security critical components.
                 \\ \hline
            T2 & Informal security policy, 
                 informal security policy model 
                 informal architectural design, 
                 informal detailed design, 
                 \\ \hline
            T3 & Informal security policy
                 semi-formal security policy model,
                 semi-formal architectural design, 
                 informal detailed design,
                 provide subset of source code for evaluation.
                 \\ \hline
            T4 & Informal security policy,
                 formal security policy model,
                 semi-formal architectural design,
                 semi-formal detailed design,
                 provide subset of source code for evaluation.
                 \\ \hline
            T5 & Informal security policy,
                 formal security policy model,
                 semi-formal architectural design,
                 semi-formal detailed design,
                 provide source code of entire product for evaluation.
                 \\ \hline
            T6 & Informal security policy,
                 formal security policy model,
                 formal architectural design,
                 semi-formal detailed design,
                 provide source code of entire product for evaluation.
                 \\ \hline
            T7 & Informal security policy,
                 formal security policy model,
                 formal architectural design,
                 formal detailed design,
                 provide source code of entire product for evaluation.
                 \\ \hline
        \end{tabular}
        \end{center}
    \caption{Summary of the CTCPEC assurance levels} \label{table:ctcpec-levels}
    \end{table}

    \subsubsection{Security Functionality Criteria}
    Unlike the TCSEC, the CTCPEC separates the functionality criteria from the assurance criteria.
    Also, unlike ITSEC, the CTCPEC does not leave the decision for the choice of security functionality 
    to the developer. The CTCPEC clearly defines a large number of security functionality criteria.
    Table~\ref{table:ctcpec-functionality} summarises the functionality criteria
    of the CTCPEC.

    As can be seen from the table, the CTCPEC contains four types of security
    functionality criteria.
        \begin{itemize}
            \item Confidentiality Criteria
            \item Integrity Criteria
            \item Availability Criteria
            \item Accountability Criteria
        \end{itemize}
    Each of these criteria types contain a number of specific criteria, each with
    a number of ratings. For example, ``Audit'' is one of the accountability
    criteria. 
    
    \begin{table}[H]
    \begin{center}
    \begin{tabular}{|>{\sffamily}p{5cm}|>{\sffamily}p{6cm}|} \hline
            \bfseries{Functionality Criteria} & \bfseries{Levels} \\ \hline\hline
            \multicolumn{2}{|l|}{\bfseries{\sffamily{Confidentiality Criteria}}} \\ \hline
            Covert Channels                 & CC-0, CC-1, CC-2, CC-3 \\ \hline
            Discretionary Confidentiality   & CD-0, CD-1, CD-2, CD-3, CD-4 \\ \hline
            Mandatory Confidentiality       & CM-0, CM-1, CM-2 \\ \hline
            Object Reuse                    & CR-0, CR-1 \\ \hline\hline

            \multicolumn{2}{|l|}{\bfseries{\sffamily{Integrity Criteria}}}   \\ \hline           
            Domain Integrity        & IB-0, IB-1, IB-2 \\ \hline
            Discretionary Integrity & ID-0, ID-1, ID-2, ID-4 \\ \hline
            Mandatory Integrity     & IM-0, IM-1, IM-2, IM-4 \\ \hline
            Physical Integrity      & IP-0, IP-1, IP-2, IP-3, IP-4 \\ \hline
            Rollback                & IR-0, IR-1, IR-2 \\ \hline
            Separation of Duties    & IS-0, IS-1, IS-2 \\ \hline
            Self Testing            & IT-0, IT-1, IT-3 \\ \hline\hline

            \multicolumn{2}{|l|}{\bfseries{\sffamily{Availability Criteria}}}   \\ \hline 
            Containment         & AC-0, AC-1, AC-2, AC-3 \\ \hline
            Fault Tolerance     & AF-0, AF-1, AF-2 \\ \hline
            Robustness          & AR-0, AR-1, AR-2, AR-3 \\ \hline
            Recovery            & AY-0, AY-1, AY-2, AY-3 \\ \hline\hline

            \multicolumn{2}{|l|}{\bfseries{\sffamily{Accountability Criteria}}}   \\ \hline
            Audit   &   WA-0, WA-1, WA-2, WA-3, WA-4, WA-5 \\ \hline
            Identification and Authentication   & WI-0, WI-1, WI-2, WI-3 \\ \hline
            Trusted Path    & WT-0, WT-1, WT-3 \\ \hline
    \end{tabular}
    \end{center}
    \caption{Summary of the CTCPEC's functionality criteria and
             available security ratings} \label{table:ctcpec-functionality}
    \end{table}

    
\subsection{Summary of Evaluation Areas}
    When evaluating systems for assurance, the evaluation occurs against a group of general criteria. 
    Table~\ref{table:areas} summarises the areas in which requirements are specified
    for each of the levels of trust in each of the documents.

    \begin{table}[H]
    \begin{center}
        \begin{tabular}{|>{\sffamily}p{4.5cm}|>{\sffamily}p{4.5cm}|>{\sffamily}p{4.5cm}|} \hline
            \bfseries{TCSEC}   &   \bfseries{ITSEC}   &   \bfseries{CTCPEC}  \\ \hline \hline
            Security Policy \newline
            Accountability  \newline
            Assurance       \newline
            Documentation
            &
            Development Process         \newline
            Development Environment     \newline
            Operational Documentation   \newline
            Operational Environment     \newline
            &
            Architecture                \newline
            Development Environment     \newline
            Development Evidence        \newline
            Operational Environment     \newline
            Security Documentation      \newline
            Security Testing            
            \\ \hline
        \end{tabular} 
    \end{center}
    \caption{Areas of Evaluation} \label{table:areas}
    \end{table}

\subsection{Equivalent Levels of Trust}
    Table~\ref{table:assurance-levels} shows the approximate correspondence between
    levels of assurance for the three documents. The relationship between the TCSEC
    classes and the ITSEC assurance levels are specified in the ITSEC~\cite{itsec}. The relationship
    between the ITSEC and the CTCPEC assurance levels, are obtained from the results of an international
    collaboration to develop a common approach to security standards described in
    \emph{Foundations for the Harmonization of Information Technology Security Standards}~\cite{harmony}.

    \begin{table}[H]
    \begin{center}
    \begin{tabular}{|>{\sffamily}p{2cm}|>{\sffamily}p{2cm}|>{\sffamily}p{2cm}|} \hline
    \bfseries{TCSEC}   &   \bfseries{ITSEC}   &   \bfseries{CTCPEC}  \\ \hline \hline
    A1      &   E6      &   T7      \\ \hline
    B3      &   E5      &   T6      \\ \hline
    B2      &   E4      &   T5      \\ \hline
    B1      &   E3      &   T4      \\ \hline
    C2      &   E2      &   T3      \\ \hline
    C1      &   E1      &   T2      \\ \hline
            &           &   T1      \\ \hline
    D       &   E0      &   T0      \\ \hline
    \end{tabular}
    \end{center}
    \caption{Comparison of Assurance Levels} \label{table:assurance-levels}
    \end{table} 

\section{Summary of General Characteristics}

    \begin{table}[H]
    \begin{center}
    \begin{tabular}{|>{\sffamily} m{7cm}|>{\sffamily}m{15mm}%
                    |>{\sffamily}m{15mm}%
                    |>{\sffamily}m{15mm}|} \hline
        \bfseries{Characteristic} & \bfseries{TCSEC} & %
        \bfseries{ITSEC} & \bfseries{CTCPEC} \\ \hline \hline

        \multicolumn{1}{|c|}{\bfseries{\sffamily{Structure}}} 
        & & & \\ \hline

        Introduction                            & YES & YES & YES \\ \hline
        Assurance Criteria                      & YES & YES & YES \\ \hline
        Separation of Functionality Criteria    & NO  & YES  & YES \\ \hline
        Additional Guidelines for Criteria      & YES & NO & YES \\ \hline
        Rationale behind Criteria               & YES & NO & YES \\ \hline
        Glossary/Definitions                    & YES & YES & YES \\ \hline 
        References/Bibliography                 & YES & YES & YES \\ \hline \hline

        \multicolumn{1}{|c|}{\bfseries{\sffamily{Levels of Trust}}} 
        & & & \\ \hline 
        
        Separation of assurance criteria from security functionality
        criteria (confidentiality, integrity, availability,
        accountability)
        & NO & YES & YES \\ \hline \hline

        \multicolumn{1}{|c|}{\bfseries{\sffamily{Security Functionality}}}
        & & & \\ \hline
        Specification of Functionality Criteria & YES & NO & YES \\ \hline \hline

        \multicolumn{1}{|c|}{\bfseries{\sffamily{Criteria Interpretations}}} 
        & & & \\ \hline 

        Reliance on separate interpretation documents to cover wider range 
        of systems & YES & NO & NO \\ \hline 

    \end{tabular}
    \end{center}
    \caption{Summary of General Characteristics} \label{table:general-char}
    \end{table}

    Table~\ref{table:general-char} gives a summary of some of the general
    characteristics of the three sets of security evaluation criteria.

    Both the ITSEC and the CTCPEC target a greater range of systems than does the TCSEC, are more explicit
    about the requirements at each assurance level and therefore do not require separate interpretation
    documents. The requirements in the TCSEC are more general and hence it is often the case that
    there are several ways to read a given statement~\footnote{Stated in the NSA's TPEP FAQ~\cite{tpepfaq}}.
    As a result, a number of official interpretations of the TCSEC have been developed.
    The interpretations are official statements articulating which of a number of possible ways to
    read a security requirement for different applications. For example, the Trusted Network Interpretation
    (TNI)~\cite{redbook} of the TCSEC, also referred to as ``The Red Book''.
    The official interpretations of the TCSEC are collectively know as the ``The Rainbow Series''~\cite{rainbow}.
 


%=========================================================================
%=========================================================================
%=========================================================================

\section{Comparing Security Functionality} 
This section compares the three sets of security evaluation criteria with 
respect to specific areas of security functionality.
The comparison is made on three important areas of security functionality. 
    \begin{itemize}
        \item Accountability
        \item Access Control
        \item Audit
    \end{itemize}

%--------------------------------------------------------------------------
\newcommand{\heading}[1]{\multicolumn{2}{|c|}{\bfseries\sffamily{#1}} \\ \hline}
\newcommand{\bc}{\begin{tabular}{|>{\sffamily}p{2cm}|>{\sffamily}p{11cm}|} \hline}
\newcommand{\ec}{\end{tabular}}
\newcommand{\approach}[1]{Approach & #1 \\ \hline}
\newcommand{\levels}[1]{Levels & #1 \\ \hline}
\newcommand{\type}[1]{Criteria Type & #1 \\ \hline}
%--------------------------------------------------------------------------

    \subsection{Accountability --- Identification and Authentication} 
    Table~\ref{table:accountability} summarises the requirements for identification and
    authentication.

    Accountability is one of four major criteria in which requirements 
    are specified for each class of the TCSEC.  
    Requirements for identification and authentication are addressed in this area. The issue of ``trusted paths'' 
    are addressed in this area for the higher classes.

    The ITSEC does not specify any criteria regarding accountability through identification and 
    authentication, as it allows
    the developer to select arbitrary security functions for a product.  
    The ITSEC recommends that ''Identification and Authentication'' is one of the categories which the developer should
    use when specifying functionality.  The ITSEC provides example functionality classes, which the 
    developer can use a guide. Identification and authentication requirements are present in all the example classes.
    A set of four levels (WI-0 to WI-3) are defined, indicating increasing security in
    terms of identifying and authenticating users.

    \begin{table}[H]
    \begin{center}
    \bc 
        \heading{TCSEC}
            \levels{Identification: C1, C2, B1, B2, B3, A1 \newline
                    Authentication: C1, C2, B1, B2, B3, A1 \newline
                    Trusted Path: B2, B3, A1 }
        \heading{ITSEC} 
            \levels{F-C1, F-C2, F-B1, F-B2, F-B3,
                    F-IN, F-AV, F-DI, F-DC, F-DX}
        \heading{CTCPEC}
            \levels{WI-0, WI-1, WI-2, WI-3}
    \ec
    \end{center}
    \caption{Comparison of Accountability Criteria --- Identification and Authentication} \label{table:accountability}
    \end{table}

%--------------------------------------------------------------------------

    \subsection{Access Control} 
    Table~\ref{table:access-control} summarises the requirements for access control
    mechanisms. In the TCSEC, access control is addressed under the area of security policy.
    The TCSEC classes look at discretionary access control and mandatory access control.
    As mentioned previously the ITSEC does not specify any security functionality.
    The ITSEC specifies that ``Access Control'' is one of the categories which the developer should
    use when specifying security functionality, and it is included in the ITSEC example functionality
    classes.
    The CTCPEC looks at access control from the perspective of Confidentiality Criteria (including
    discretionary and mandatory confidentiality) and Integrity Criteria (including
    discretionary and mandatory integrity).

    \begin{table}[H]
    \begin{center}
    \bc
        \heading{TCSEC}
            \levels{Discretionary Access Control: C1, C2, B1, B2, B3, A1 \newline
                    Mandatory Access Control: B1, B2, B3, A1}
        \heading{ITSEC}
            \levels{F-C1, F-C2, F-B1, F-B2, F-B3,
                    F-IN, F-AV, F-DI, F-DC, F-DX}
        \heading{CTCPEC}
            \levels{Discretionary Confidentiality: CD-0, CD-1, CD-2, CD-3 CD-4 \newline
                    Mandatory Confidentiality: CM-0, CM-1, CM-2 \newline
                    Discretionary Integrity: ID-0, ID-1, ID-2, ID-4 \newline
                    Mandatory Integrity: IM-0, IM-1, IM-2, IM-4}
    \ec
    \end{center}
    \caption{Comparison of Access Control Criteria} \label{table:access-control}
    \end{table}

%--------------------------------------------------------------------------
    \subsection{Audit} 
    Table~\ref{table:audit} summarises the requirements for auditing. Audit requirements
    in the TCSEC are covered under accountability criteria and are present in classes
    above class D. The ITSEC recommends that ``Audit'' be one of the categories when
    specifying security functionality. Like the TCSEC, the CTCPEC addresses audit under
    accountability criteria but specifies a number of audit levels.

    \begin{table}[H]
    \begin{center}
    \bc
        \heading{TCSEC}
            \levels{C2, B1, B2, B3, A1}
        \heading{ITSEC}
            \levels{F-C1, F-C2, F-B1, F-B2, F-B3,
                    F-IN, F-AV, F-DI, F-DC, F-DX} 
        \heading{CTCPEC}
            \levels{WA-0, WA-1, WA-2, WA-3, WA-4, WA-5}
    \ec
    \end{center}
    \caption{Comparison of Audit Criteria} \label{table:audit}
    \end{table}

%--------------------------------------------------------------------------

\section{Comparing Security Functionality Assurance} 
    This section compares the security evaluation criteria on the basis
    of functionality assurance. The comparison is made against seven
    system engineering characteristics which are important in
    assuring that a system is secure. The characteristic are:
    \begin{itemize}
        \item Security Policy
        \item System Design
        \item Implementation
        \item Security Testing
        \item Security Documentation
        \item Configuration Management
    \end{itemize}

%--------------------------------------------------------------------------
    \subsection{Security Policy} 
    A system's security policy is a set of security requirements which 
    a system must satisfy. A detailed description of a system's security 
    policy is known as the security policy model. The TCSEC, the ITSEC and the CTCPEC 
    all address the issues of security policies and security policy models very similarly.
    \begin{table}[H]
    \begin{center}
    \bc
        \heading{TCSEC}
            \approach{Security Policy is one of the four major criteria addressed
                      for each class of the TCSEC}
            \levels{C1, C2, B1, B2, B3, A1}
        \heading{ITSEC}
            \approach{For each assurance level in the ITSEC, security policy
                      is assessed under the area of ``Development Process - Requirements''}
            \levels{E1, E2, E3, E4, E5, E6, E7}
        \heading{CTCPEC}
            \approach{For each assurance level, the CTCPEC addresses the issue of 
                      security policy under the area of the system's ``Architecture''}
            \levels{T1, T2, T3, T4, T5, T6, T7}
    \ec
    \end{center}
    \caption{Comparison of Security Policy Criteria} \label{table:policy}
    \end{table}
    Table~\ref{table:policy} summarises the requirements for security policy for each of the documents.
    In general for the lower level of assurances in the TCSEC, the ITSEC and the CTCPEC, the
    requirements include:
    \begin{itemize}
        \item Informal security policy
        \item Informal or semi-formal security policy model
        \item Tracing between the security policy and the security policy model
        \item Tracing between the security policy model and the architectural design
    \end{itemize}
    At the higher levels of assurance, the requirements are stricter and include:
    \begin{itemize}
        \item Informal security policy
        \item Formal security policy model
        \item Demonstration of security policy mapping to security policy model
        \item Formal verification that the security policy model maps to the architectural design
    \end{itemize}

%--------------------------------------------------------------------------

    \subsection{System Design} 
    Table~\ref{table:design} shows which assurance levels of the three documents
    require assurance through the provision of a system design. As can be seen
    from the table, evidence for system design is a requirement for all classes
    of the TCSEC above class D, all assurance levels of the ITSEC above E0,
    and all assurance levels of the CTCPEC above T0.
    \begin{table}[H]
    \begin{center}
    \bc
        \heading{TCSEC}
            \levels{C1, C2, B1, B2, B3, A1}
        \heading{ITSEC}
            \levels{E1, E2, E3, E4, E5, E6}
        \heading{CTCPEC}
            \levels{T1, T2, T3, T4, T5, T6, T7}
    \ec
    \end{center}
    \caption{Comparison of System Design Criteria} \label{table:design}
    \end{table}
    System design in the TCSEC is addressed under the ``Assurance'' category for each class.
    In the ITSEC, the system design is addressed under the ``Development Process Criteria''
    in the areas of ``Architectural Design'' and ``Detailed Design''. The CTCPEC addresses
    system design under the criteria for ``Development Evidence''.
    
    The requirements for system design across the three sets of security evaluation criteria
    are very similar. All the documents have requirements for:
    \begin{itemize}
        \item Architectural design
        \item Detailed design
        \item Mapping between security policy model and architectural design
        \item Mapping between the architectural design and the detailed design
        \item Mapping between the detailed design and the implementation
    \end{itemize}
    At the higher levels of assurance, there is a greater emphasis on formal design and 
    verification techniques.

%--------------------------------------------------------------------------

    \subsection{Implementation} 
    Implementation issues in the TCSEC are addressed under the Assurance Criteria, in the areas dealing
    with operational assurance and development life-cycle assurance. However, the TCSEC does not 
    go into much detail regarding implementation, focusing more on verification and testing techniques
    to show that the system satisfies it's security policy. The TCSEC relies on the the interpretation
    documents, the ``Rainbow Series''~\cite{rainbow}, for specific requirements regarding implementation
    for different types of systems.

    The ITSEC and the CTCPEC on the other hand, have more specific implementation requirements, at each
    level of assurance. In each assurance level, the ITSEC addresses implementation requirements in the
    following criteria groups.
    \begin{itemize}
        \item Construction --- The Development Process
            \begin{itemize}
                \item Implementation
            \end{itemize}
        \item Construction --- The Development Environment
            \begin{itemize}
                \item Programming Languages and Compilers
                \item Developer's Security
            \end{itemize}
    \end{itemize}
    The CTCPEC addresses implementation requirements, in the following criteria groups.
    \begin{itemize}
        \item Development Environment
        \item Development Evidence
        \item Operational Environment
    \end{itemize}
    In all the security evaluation criteria being compared, in the higher assurance levels,
    there is an emphasis on semi-formal and formal verification techniques to show the mapping
    between the detailed design and the requirement. Also common in the higher assurance levels,
    is the requirement that security critical source code or all the source code be provided
    when the system is being evaluated.

%--------------------------------------------------------------------------

    \subsection{Security Testing} 
    The TCSEC, the ITSEC and the CTCPEC have very similar requirements
    for security testing. All assurance levels (above the non-compliant
    levels) in all the documents address the issue of security testing.
    The emphasis is on:
    \begin{itemize}
        \item Test Plan Documentation
        \item Test Results Documentation
        \item Evidence of Testing
        \item Justification that coverage of testing is sufficient
        \item Evidence using test reports to show that the system
              satisfies it's security policy.
    \end{itemize}

%--------------------------------------------------------------------------

    \subsection{Security Documentation} 
    The requirements for security documentation are very similar in the TCSEC,
    the ITSEC, and the CTCPEC. All documents require that the system is provided
    with security documentation for the user and for the system administrator
    at all levels of assurance. This is summarised in table~\ref{table:docs}.
        \begin{table}[H]
        \bt
        \cmp{Security Features User's Guide \newline (User Documentation)}
            {C1, C2, B1, B2, B3, A1}
            {E1, E2, E3, E4, E5, E6}
            {T1, T2, T4, T4, T5, T6, T7}
        \cmp{Trusted Facility Manual \newline (Administrator's Documentation)}
            {C1, C2, B1, B2, B3, A1}
            {E1, E2, E3, E4, E5, E6}
            {T1, T2, T3, T4, T5, T6, T7}
        \et
        \caption{Comparison of ecurity Documentation Criteria} \label{table:docs}
        \end{table}
    Security documentation requirements are addressed in the ``Documentation''
    criteria for each class in the TCSEC, in the ``Operational Documentation''
    criteria of the ITSEC, and in the ``Security Documentation''
    criteria of the CTCPEC.

%--------------------------------------------------------------------------

    \subsection{Configuration Management}
        The use of a configuration control system throughout all phases of the
        system's development is requirement at all levels of assurance for
        the TCSEC, the ITSEC and the CTCPEC. This is summarised in table~\ref{table:config}.
        \begin{table}[H]
        \bt
        \cmp{Configuration Control System}
            {C1, C2, B1, B2, B3, A1}
            {E1, E2, E3, E4, E5, E6}
            {T1, T2, T3, T4, T5, T6, T7}
        \et
        \caption{Comparison of Configuration Management Criteria} \label{table:config}
        \end{table}
        Configuration management is addressed under ``Assurance'' in the TCSEC,
        and under ``Development Environment'' in the both the ITSEC and the CTCPEC.

