\chapter{Conclusions}

The choice of which security evaluation criteria will be targeted by a
developer depends on a number of different factors. Since conformance to
a standard is important, the choice of criteria is likely to be limited
to the TCSEC from the USA, the ITSEC from Europe or the CTCPEC from Canada, 
as these are the most widely used and accepted security evaluation criteria.

The factor most influencing the choice of a target set of security
evaluation criteria for a trusted product is the product's intended
market.

For example If a product is being developed with government or military customers in
mind, it is in the the best interest of the developer to choose the
evaluation criteria which has been specified by the government of the
nation which the product is targeted at. In the United States, the
National Security Agency requires conformance to the TCSEC, the ITSEC in
Europe and the CTCPEC in Canada. Countries which do not have their own
evaluation criteria usually adopt either the TCSEC, the ITSEC or the
CTCPEC. For example in Australia, the Defence Signals Directorate (DSD)~\cite{dsd} 
has adopted the ITSEC for use by the Australian government.

Choosing a set of security evaluation criteria as a developer is usually
an easier job than a government or commercial organisation choosing
a particular evaluation criteria for which products which they use must
conform to. This is simply because it is often the case for the
developer that the choice has already been made by the client.

From a security functionality and security assurance perspective the
three sets of evaluation criteria compared, specify very similar
requirements. Regardless of the whether the choice is the TCPEC, the
ITSEC or the CTCPEC, building a trusted system does require important
security issues to be taken into account.
These include accountability, access control, audit and the development of 
security policy models, as well as the use of well defined and rigid
software engineering practices such as security requirements
specification, architectural and detailed design, secure implementation, 
security testing, the production of security documentation, 
configuration management, and at the
higher levels of assurance, the use of formal verification techniques to
ensure that the system being built satisfies the security policy model
throughout all the phases of the development process.

The many similarities between the TCSEC, the ITSEC and the CTCPEC is to
be expected because the European and Canadian documents were
originally based on the US experience with the TCSEC, which has been in
use since 1983.

Based on the comparison of the three sets of security evaluation
criteria, from an operational perspective the ITSEC and the CTCPEC are
to be preferred over the TCSEC. The main reason for this is that the
criteria in the ITSEC and the CTCPEC are more wide reaching, take more
cases into account and allow the targeting of a more diverse range of
systems without the reliance on separate interpretation documents as is
the case with the TCSEC. 

Selecting between the ITSEC and the CTCPEC
depends on the requirements of the particular organisation. 
Both the ITSEC and the CTCPEC separate functionality criteria from assurance
criteria. The CTCPEC contains detailed and very specific security functionality
criteria. On the other hand, the ITSEC allows a system to have any security
function which the developer defines.

Therefore, if a choice was to be made between the three compared
documents, (not taking external requirements such as those of a client or
the wide spread use of a document into account), it would have to be the
ITSEC or the CTCPEC. Both standards do not require external interpretations,
they separate functionality and assurance criteria, and both apply to wide
range of systems. A choice between them depends on the developer. 
It is up to the developer to choose the flexibility of the ITSEC in specifying
security functionality or the rigidness of the CTCPEC, where security functions
are explicitly defined and have their own level of trust separate from the assurance levels.

The increase in the use of computer systems for all sorts of
applications, has resulted in the requirement for the existence of 
secure systems which have been evaluated against a set of
well defined criteria and can be rated to provide a specific level of
trust. Trusted systems, once only required by the military now are in
wide spread use in government, industry, commerce, education and many
other areas. Hence, the application of security evaluation criteria 
to evaluate systems is becoming increasingly important, and is very much
an active area of research and development in the large domain of
computer security. This world wide effort has resulted in the current development of the
Common Criteria combining the TCSEC, the ITSEC and CTCPEC into one
standard set of security evaluation criteria, which will in the future
will provide a common base from which a trusted system can be evaluated.

